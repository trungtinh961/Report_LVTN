Trong nền công nghiệp hiện đại 4.0 hiện nay, các doanh nghiệp hoạt động với dây chuyền sản xuất thực hiện thủ công và quản lí doanh nghiệp thông qua sổ sách viết tay có thể sẽ bị quá tải với khối lượng công việc và hàng hóa khổng lồ, khó bắt kịp các đối thủ của mình. Để tiếp tục phát triển và cạnh tranh, các doanh nghiệp đòi hỏi phải thay đổi quy trình hoạt động và quản lí của mình từ thủ công sang các hệ thống tự động, sử dụng các phần mềm quản lí chuyên dụng. \par

Hiện nay, có nhiều doanh nghiệp lựa chọn số hóa toàn bộ quy trình hoạt động của mình, bên cạnh đó cũng có doanh nghiệp lựa chọn số hóa một phần quy trình hoạt động, ưu tiên một số dây chuyền quan trọng trước, vừa phù hợp với kinh phí của doanh nghiệp nhưng cũng có thể đảm bảo có thể cạnh tranh với các doanh nghiệp đối thủ. \par

Với tinh thần trên, nhóm thực hiện quyết định xây dựng một hệ thống "Quản lý quá trình gia công nhuộm", nhằm mục đích đáp ứng một số nhu cầu của các doanh nghiệp sản xuất vải sợi bao gồm: quản lý các cây vải mộc của doanh nghiệp, quản lí các đơn đặt hàng của doanh nghiệp, quản lý các cây vải thành phẩm sau khi nhuộm, quản lý các cây vải đang tồn kho ở các xưởng gia công, quản lý công nợ của các xưởng gia công, và quản lý các cây vải lỗi trả lại, thanh toán công nợ, thống kê quá trình gia công nhuộm của doanh nghiệp, quản lí các xưởng gia công nhuộm. \par

Đề tài được xây dựng trên nền tảng web, một nền tảng đã trở nên rất phổ biến với nhiều người hiện nay. Với những chức năng được hiện thực hứa hẹn sẽ giúp cho doanh nghiệp vải sợi có thể sử dụng và gia tăng hiệu quả kinh doanh của mình.

Luồng hoạt động chính của ứng dụng bắt đầu bằng việc doanh nghiệp xuất vải mộc chuyển đến các xưởng nhuộm. Khi có nhu cầu muốn nhuộm vải, doanh nghiệp sẽ tạo một đơn đặt hàng và chuyển đến xưởng nhuộm. Xưởng nhuộm sau khi nhận được yêu cầu đặt hàng sẽ lấy vải mộc đã nhập từ trước đem ra nhuộm tạo thành cây vải thành phẩm. Việc nhuộm vải này sẽ thực hiện theo từng lô nhuộm, sau khi nhuộm xong từng lô sẽ tiến hành gửi vải thành phẩm về lại doanh nghiệp. Doanh nghiệp tiến hành nhập vải thành phẩm, chi phí cho mỗi đợt nhập vải thành phẩm này được tính theo số lượng vải thành phẩm, các thông số của vải, và theo giá của từng xưởng nhuộm khác nhau - chi phí này được gọi là công nợ). Doanh nghiệp  thanh toán công nợ cho xưởng nhuộm ở khoảng thời gian nào đó tùy ý, chỉ dựa vào tổng công nợ của xưởng nhuộm đó chứ không phụ thuộc vào các lần nhập vải thành phẩm. Ngoài ra, nếu doanh nghiệp phát hiện cây vải thành phẩm có bị lỗi sẽ có thể trả lại cho xưởng nhuộm, khi này sẽ cập nhật lại công nợ của xưởng nhuộm (trừ đi giá của khúc cây vải trả).

Dự án được thực hiện bởi nhóm sinh viên tại trường Đại học Bách Khoa thành phố Hồ Chí Minh.  \par